%%%%%%%%%%%%%%%%%%%%%%%%%%%%%%%%%%%%%%%%%%%%%%%%%%%%%%%%%%%%%%%%%%%%%%
%
% Author: Jan Tulak, xtulak00
%
%%%%%%%%%%%%%%%%%%%%%%%%%%%%%%%%%%%%%%%%%%%%%%%%%%%%%%%%%%%%%%%%%%%%%%

%%%%%%%%%%%%%%%%%%%%%%%%%%%%%%%%%%%%%%%%%%%%%%%%%%%%%%%%%%%%%%%%%%%%%%
\section{Úvod}
Navzdory stále rostoucímu významu open-source\footnote{TODO sources: microsoft opening, apple based on, millenials and jobs...} v ekonomice, je obtížné nalézt články či výzkumy zaobírající se studiem open-source z pohledu teorie her. A to jak z pohledu ekonomického (soupeření open-source a proprietárních produktů či dodavatelů), tak z pohledu spíše psychologického (pohled na jednotlivé členy open-source komunity, jejich motivaci, očekávaný zisk, a podobně.)

V této zprávě se tedy pokouším shrnout dostupné materiály k tématu, které se mi podařilo nalézt.

%%%%%%%%%%%%%%%%%%%%%%%%%%%%%%%%%%%%%%%%%%%%%%%%%%%%%%%%%%%%%%%%%%%%%%
\section{Open-source vs proprietární software}
Podle dotazníku provedeného Computer World v roce 2004 byl "vendor lock-in"\footnote{{\em Vendor lock-in} popisuje situaci, kdy zákazník je nucen pokračovat v užívání konkrétního produktu/dodavatele i při existenci alternativ, protože přechod na jiné řešení či jiného dodavatele by byl pro zákazníka velmi nákladný. Možností, jak může dodavatel lock-in dosáhnout je vícero: (1) systém je navržen tak, aby nebyl kompatibilní s alternativními řešeními od ostatních dodavatelů; (2) využíváním uzavřených architektur a proprietárních standardů; (3) Licenční a patentovou politikou.} jednou z hlavních obav mezi IT odborníky ohledně nastupujícího trendu cloud computingu~\cite{computer-world-2004}. Ve výzkumu provedeném North Bridge v roce 2015 je snaha vyhnout se vendor lock-in zmiňována jako jeden z důvodů pro výběr open-source řešení~\cite{survey-2015}.



%%%%%%%%%%%%%%%%%%%%%%%%%%%%%%%%%%%%%%%%%%%%%%%%%%%%%%%%%%%%%%%%%%%%%%
\section{Shrnutí}







%%%%%%%%%%%%%%%%%%%%%%%%%%%%%%%%%%%%%%%%%%%%%%%%%%%%%%%%%%%%%%%%%%%%%%
\section{Zdroje}
%\bibliographystyle{czechiso}
\bibliographystyle{plain}
%\begin{flushleft}
\bibliography{proj}
%\end{flushleft}

%\section{Přílohy}
